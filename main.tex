\documentclass{beamer}

\usepackage[utf8]{inputenc}
\usepackage[T1]{fontenc}
\usepackage{lmodern}
\usepackage[scale=2]{ccicons}
\usepackage{tikz}

\usepackage[english]{babel}
% \usepackage[english, polish]{babel}

\usepackage{listings}
\lstset{basicstyle=\ttfamily\footnotesize,breaklines=true}

\usepackage{siunitx}
\usepackage{pifont}
\usepackage{amsmath,amssymb,amsfonts}
\usepackage{graphicx}
\usepackage[export]{adjustbox}
\usepackage{float}
\usepackage{xcolor}
\usepackage{xspace}
\usepackage{setspace}

\usetheme{CambridgeUS}
\useoutertheme{miniframes}
\usecolortheme{dolphin}
\usefonttheme{professionalfonts}
\beamertemplatenavigationsymbolsempty

\setbeamertemplate{background}{
    \tikz[overlay,remember picture]\node[opacity=0.09]at (current page.center){
        \includegraphics[width=7cm]{images/WF_sygnet_czarny.png}
    };
}

\defbeamertemplate{headline}{pwtemplate}{
    \leavevmode
    \hbox{
        \begin{beamercolorbox}[wd=\paperwidth,ht=2.25ex,dp=1ex,right]{title in head/foot}
            \usebeamerfont{title in head/foot}
            \insertframenumber/\inserttotalframenumber\hspace*{2em}
        \end{beamercolorbox}}
    \vskip0pt
}

\defbeamertemplate{footline}{pwtemplate}{ % gray!0 is white
    \leavevmode
    \setbeamercolor{lfooterbox}{bg=gray!0}
    \begin{beamercolorbox}[wd=.33\paperwidth,ht=7pt,dp=4pt,center]{lfooterbox}
        % \usebeamerfont{author in head/foot}
        \includegraphics[scale=0.07]{images/WF_PW_sygnet_EN_czarny_RGB.png}
    \end{beamercolorbox}%
    \setbeamercolor{cfooterbox}{bg=gray!0}
    \begin{beamercolorbox}[wd=.34\paperwidth,ht=7pt,dp=4pt,center]{cfooterbox}
        % \usebeamerfont{date in head/foot}
    \end{beamercolorbox}%
    \setbeamercolor{rfooterbox}{bg=gray!0}
    \begin{beamercolorbox}[wd=.33\paperwidth,ht=7pt,dp=4pt,right]{rfooterbox}
        \usebeamerfont{title in head/foot}
        \insertshortauthor \hspace*{2em}
    \end{beamercolorbox}
}



\newcommand{\todo}[1]{\textcolor{red}{TODO: #1}}
\newcommand{\keV}[0]{\si{\kilo\electronvolt}}
\newcommand{\MeV}[0]{\si{\mega\electronvolt}}
\newcommand{\GeV}[0]{\si{\giga\electronvolt}}

\newcommand{\aegis}{AE$\overline{\textrm{g}}$IS\xspace}

\newcommand{\imagesource}[1]{
    \begin{spacing}{0.5}
        \texttt{\textit{ \tiny{source: #1}}}
    \end{spacing}
}

\newenvironment{columnframe}[1]{
    \begin{frame}[environment=columnframe,fragile]{#1}
        \begin{columns}
            }{
        \end{columns}
    \end{frame}
}





\title{Characterization of interaction of gamma radiation }
    \subtitle{\textit{with BGO scintillators}}
\author[T. Fic]{Tobiasz Fic\\ \vspace{5pt} \footnotesize Promotor: Dr Georgy Kornakov}
\institute[WUT]{Faculty of Physics, Warsaw University of Technology}

\date{11 June 2025}

\begin{document}

\setbeamertemplate{headline}{}
\setbeamertemplate{footline}{}
\begin{frame}
    \maketitle
\end{frame}

\setbeamertemplate{background}{}
\setbeamertemplate{headline}[pwtemplate]
\setbeamertemplate{footline}[pwtemplate]

\begin{columnframe}{CERN and AEgIS}
    \begin{column}{0.5\textwidth}
        \begin{figure}
            \centering
            \includegraphics[width=0.6\textwidth, frame]{images/pusheen.png}
        \end{figure}
        \todo{Picture of AEgIS experiment}
    \end{column}
    \begin{column}{0.5\textwidth}
        \begin{itemize}
            \item \todo{what aegis does}
        \end{itemize}
    \end{column}
\end{columnframe}

\begin{columnframe}{Background and Motivation}
    \begin{column}{0.5\textwidth}
        \begin{figure}
            \centering
            \includegraphics[width=0.6\textwidth, frame]{images/pusheen.png}
        \end{figure}
        \todo{Figure of antiprotonic atom}
    \end{column}
    \begin{column}{0.5\textwidth}
        \todo{What do we want to measure?}
    \end{column}
\end{columnframe}

\begin{columnframe}{The ideal scintillator detector}
    \begin{column}{0.5\textwidth}
        \begin{figure}
            \centering
            \includegraphics[width=0.6\textwidth, frame]{images/pusheen.png}
        \end{figure}
        \todo{Two images: diagram of scintillator detector and a spectrum of a gamma source}
    \end{column}
    \begin{column}{0.5\textwidth}
        \begin{itemize}
            \item Energy of incident gamma photon is turned into multiple UV photons
            \item UV photons are collected by the PMT
            \item The signal from the PMT is perfectly proportional in height to the energy of  the incident gamma photon
        \end{itemize}
    \end{column}
\end{columnframe}

\begin{columnframe}{Photonis PET detectors}
    \begin{column}{0.5\textwidth}
        \begin{figure}
            \centering
            \includegraphics[width=0.6\textwidth, frame]{images/pusheen.png}
        \end{figure}
        \todo{Picture of PMT}
    \end{column}
    \begin{column}{0.5\textwidth}
        \todo{describe what we have and what it is}
    \end{column}
\end{columnframe}

\begin{columnframe}{Expected results}
    \begin{column}{0.5\textwidth}
        \begin{figure}
            \centering
            \includegraphics[width=0.6\textwidth, frame]{images/pusheen.png}
        \end{figure}
        \todo{Picture of three sources}
    \end{column}
    \begin{column}{0.5\textwidth}
        \begin{figure}
            \centering
            \includegraphics[width=0.6\textwidth, frame]{images/pusheen.png}
        \end{figure}
        \todo{Their expected spectra}
    \end{column}
\end{columnframe}

\begin{columnframe}{Teardown}
    \begin{column}{0.5\textwidth}
        \begin{figure}
            \centering
            \includegraphics[width=0.6\textwidth, frame]{images/pusheen.png}
        \end{figure}
        \todo{Teardown sequence}
    \end{column}
    \begin{column}{0.5\textwidth}
        \begin{itemize}
            \item The initial performance of the module was subpar
            \item The detector was disassembled for component-level analysis
        \end{itemize}
    \end{column}
\end{columnframe}

% \begin{columnframe}{}
%     \begin{column}{0.5\textwidth}
%     \end{column}
%     \begin{column}{0.5\textwidth}
%     \end{column}
% \end{columnframe}

\begin{columnframe}{PMT noise to signal ratio}
    \begin{column}{0.5\textwidth}
        \begin{figure}
            \centering
            \includegraphics[width=0.6\textwidth, frame]{images/pusheen.png}
        \end{figure}
        \todo{plot of PMT noise to signal ratio}
    \end{column}
    \begin{column}{0.5\textwidth}
        \begin{itemize}
            \item The noise of the PMT was measured as a function of bias voltage
            \item The signal height was measured separately
            \item The noise to signal ratio was calculated
            \item Optimal bias voltage was found to be 1450 \si{\volt}
        \end{itemize}
    \end{column}
\end{columnframe}

\begin{columnframe}{Signal/Bias Voltage Time Characteristics}
    \begin{column}{0.5\textwidth}
        \begin{figure}
            \centering
            \includegraphics[width=0.6\textwidth, frame]{images/pusheen.png}
        \end{figure}
        \todo{plot signal rise/fall time fitting}
    \end{column}
    \begin{column}{0.5\textwidth}
        \begin{figure}
            \centering
            \includegraphics[width=0.6\textwidth, frame]{images/pusheen.png}
        \end{figure}
        \todo{two plots one on top of the other: stable rise times and stable fall times}
    \end{column}
\end{columnframe}

\begin{columnframe}{Signal Light Intensity Time Characteristics}
    \begin{column}{0.5\textwidth}
        \begin{figure}
            \centering
            \includegraphics[width=0.6\textwidth, frame]{images/pusheen.png}
        \end{figure}
        \todo{plot signal rise/fall time fitting}
    \end{column}
    \begin{column}{0.5\textwidth}
        \begin{figure}
            \centering
            \includegraphics[width=0.6\textwidth, frame]{images/pusheen.png}
        \end{figure}
        \todo{two plots one on top of the other: stable rise times and stable fall times}
    \end{column}
\end{columnframe}

\begin{columnframe}{PMT Spectroscopy Setup}
    \begin{column}{0.5\textwidth}
    \end{column}
    \begin{column}{0.5\textwidth}
    \end{column}
\end{columnframe}

\begin{columnframe}{PMT Spectroscopy Raw Results}
    \begin{column}{0.5\textwidth}
        \todo{results here}
    \end{column}
    \begin{column}{0.5\textwidth}
        \todo{describe and remind what we are looking for}
    \end{column}
\end{columnframe}

\begin{columnframe}{PMT Spectroscopy Results}
    \begin{column}{0.5\textwidth}
    \end{column}
    \begin{column}{0.5\textwidth}
    \end{column}
\end{columnframe}

\begin{columnframe}{SiPM Spectroscopy Setup}
    \begin{column}{0.5\textwidth}
    \end{column}
    \begin{column}{0.5\textwidth}
    \end{column}
\end{columnframe}

\begin{columnframe}{SiPM Spectroscopy Results}
    \begin{column}{0.5\textwidth}
    \end{column}
    \begin{column}{0.5\textwidth}
    \end{column}
\end{columnframe}

\begin{columnframe}{Lastly... a Muon Telescope}
    \begin{column}{0.5\textwidth}
        \begin{figure}
            \centering
            \includegraphics[width=0.6\textwidth, frame]{images/pusheen.png}
        \end{figure}
        \todo{Picture of setup}
    \end{column}
    \begin{column}{0.5\textwidth}
        \todo{plot of landau}
    \end{column}
\end{columnframe}

% \begin{columnframe}{}
%     \begin{column}{0.5\textwidth}
%     \end{column}
%     \begin{column}{0.5\textwidth}
%     \end{column}
% \end{columnframe}

\setbeamertemplate{headline}{}
\setbeamertemplate{footline}{}
\begin{frame}{}
    \centering
    \Large{Thank you for your attention}
\end{frame}


\end{document}